
\chapwithtoc{Introduction}


Expected length of the introduction is between 1--4 pages. Longer introductions may require sub-sectioning with appropriate headings --- use \texttt{\textbackslash{}section*} to avoid numbering (with section names like `Motivation' and `Related work'), but try to avoid lengthy discussion of anything specific. Any ``real science'' (definitions, theorems, methods, data) should go into other chapters.
\todo{You may notice that this paragraph briefly shows different ``types'' of `quotes' in TeX, and the usage difference between a hyphen (-), en-dash (--) and em-dash (---).}

It is very advisable to skim through a book about scientific English writing before starting the thesis. I can recommend `\citetitle{glasman2010science}' by \citet{glasman2010science}.


\section{Motivation}

I feel like starting this thesis with the introduction into skyrmions. Lets keep it very non-specific just say what they are, why are they interesting and what are their potential use in spintronics and SNN and ANN \citet{chen2020nanoscale}...

\section{What is the nature of the problem the thesis is addressing?}

We are building a tool for 


The deeplearning models are often look upon as black boxes, but it doesn't have to be that way. We choose relatively CNN architecture with a clear interpretability and we will study the effects of different choises in both model architecture and it's training. Therefore this thesis won't only give us a solution for the problem of classification to different magnetic phases, but also provides a general insight into how different choices effect the model we are using. With such insight we are then able to make rational choices in model architecture and have more power in interpretation of the results. 

\section{What is the common approach for solving that problem now?}

Convolutional neural networks.

\section{How this thesis approaches the problem?}

We are exploring different ways of regularization in order to help the neural network to learn the specific things that we want from her. Namely we want to improve the classification around phase changes where the images contains a combination of two different phases.

\section{What are the results? Did something improve?}

We will see

\section{What can the reader expect in the individual chapters of the thesis?}



\todo{think about what are the necessary chapters}
Chapters
\begin{itemize}
	\item What are skyrmions why they are interesting
	\item what are neural networks and how they work, how are they trained
	\item What is regularization 
		
\end{itemize}

