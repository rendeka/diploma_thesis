\chapwithtoc{Introduction}


\section{What are skyrmions}
\citet{chen2020nanoscale}

\todo{briefly about history}
The term skyrmion originates from the particle theory in 1961, when Tony Skyrme introduced it as a model of the nucleon. Although this model turned out not to be correct, skyrmions as topological objects, emerged later in many other fields. In this thesis we will be dealing with \emph{magnetic skyrmions}. They are topologically stable field configurations with particle-like properties. Magnetic skyrmions were first experimentally discovered in 2009 \cite{mühlbauer2009skyrmion} in magnetic materials. Magnetic skyrmions are characteristic by rotationaly symmetric, continuosly changing spin vectors from it's middle towards it's edges.

\todo{where can we find them}

\todo{potential applications in spintronics}



\section{Problem statement}
\todo{classification for continuous phase transitions}

\todo{Approach from article Mazurenko}

\todo{How do we want to improve on Mazurenko}
\citet{iakovlev2018supervised} has used the simplest machine learning architecture. With the rise of deeplearning in recent years and development of different techniques to improve the statistical models, we have many options how to try to improve on this basic architecture.



\section{How this thesis approaches the problem?}

The machine learning models are often looked upon as black boxes, but it doesn't have to be that way. We choose relatively CNN architecture with a clear interpretability and we will study the effects of different choises in both model architecture and it's training. Therefore this thesis won't only give us a solution for the problem of classification to different magnetic phases, but also provides a general insight into how different choices effect the model we are using. With such insight we are then able to make rational choices in model architecture and have more power in interpretation of the results.

We are exploring different ways of regularization in order to help the neural network to learn the specific things that we want from her. Namely we want to improve the classification around phase changes where the images contains a combination of two different phases.

\section{What are the results? Did something improve?}

We will see

\section{Overview of next chapters}

In \ref{chap1} we will properly introduce magnetic skyrmions with the physical background. We will continue by description an element definning our problem, which is our dataset.

In \ref{chap2} we will explain important machine learning concepts and define model architectures used in our numerical experiments. Then we bring special attention to different kinds of regularization techniques used in our models.

In \ref{chap3} we present the results comparing different model architectures and different regularization techniques.

