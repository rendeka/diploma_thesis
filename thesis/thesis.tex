\documentclass[12pt,a4paper,twoside,openright]{report}
\let\openright=\cleardoublepage



%%% Choose a language %%%

\newif\ifEN
\ENtrue   % uncomment this for english
%\ENfalse   % uncomment this for czech

%%% Configuration of the title page %%%

\def\ThesisTitleStyle{mff} % MFF style
%\def\ThesisTitleStyle{cuni} % uncomment for old-style with cuni.cz logo
%\def\ThesisTitleStyle{natur} % uncomment for nature faculty logo

\def\UKFaculty{Faculty of Mathematics and Physics}
%\def\UKFaculty{Faculty of Science}

\def\UKName{Charles University in Prague} % this is not used in the "mff" style

% Thesis type names, as used in several places in the title
%\def\ThesisTypeTitle{\ifEN BACHELOR THESIS \else BAKALÁŘSKÁ PRÁCE \fi}
\def\ThesisTypeTitle{\ifEN MASTER THESIS \else DIPLOMOVÁ PRÁCE \fi}
%\def\ThesisTypeTitle{\ifEN RIGOROUS THESIS \else RIGORÓZNÍ PRÁCE \fi}
%\def\ThesisTypeTitle{\ifEN DOCTORAL THESIS \else DISERTAČNÍ PRÁCE \fi}
\def\ThesisGenitive{\ifEN bachelor \else bakalářské \fi}
%\def\ThesisGenitive{\ifEN master \else diplomové \fi}
%\def\ThesisGenitive{\ifEN rigorous \else rigorózní \fi}
%\def\ThesisGenitive{\ifEN doctoral \else disertační \fi}
\def\ThesisAccusative{\ifEN bachelor \else bakalářskou \fi}
%\def\ThesisAccusative{\ifEN master \else diplomovou \fi}
%\def\ThesisAccusative{\ifEN rigorous \else rigorózní \fi}
%\def\ThesisAccusative{\ifEN doctoral \else disertační \fi}



%%% Fill in your details %%%

% (Note: \xxx is a "ToDo label" which makes the unfilled visible. Remove it.)
\def\ThesisTitle{Effect of regularization on classification of magnetic topological phases}
\def\ThesisAuthor{Andrej Rendek}
\def\YearSubmitted{2025}

% department assigned to the thesis
\def\Department{Department of Condensed Matter Physics}
% Is it a department (katedra), or an institute (ústav)?
\def\DeptType{Department}

\def\Supervisor{RNDr. Pavel Baláž, Ph.D.}
\def\SupervisorsDepartment{Department of Condensed Matter Physics}

% Study programme and specialization
\def\StudyProgramme{Mathematical and Computational Modelling in Physics}
\def\StudyBranch{FMPMP}

\def\Dedication{%
Dedication. \xxx{It is nice to say thanks to supervisors, friends, family, book authors and food providers.}
}

\def\AbstractEN{%
\xxx{Abstracts are an abstract form of art. Use the most precise, shortest sentences that state what problem the thesis addresses, how it is approached, pinpoint the exact result achieved, and describe the applications and significance of the results. Highlight anything novel that was discovered or improved by the thesis. Maximum length is 200 words, but try to fit into 120. Abstracts are often used for deciding if a reviewer will be suitable for the thesis; a well-written abstract thus increases the probability of getting a reviewer who will like the thesis.}
% ABSTRACT IS NOT A COPY OF YOUR THESIS ASSIGNMENT!
}

\def\AbstractCS{%
\xxx{You will need to submit both Czech and English abstract to the SIS, no matter what language you use in the thesis. If writing in English, translate the contents of \texttt{\textbackslash{}AbstractEN} into this field. In case you do not speak czech, your supervisor should be able to help you with the translation.}
}

% 3 to 5 keywords (recommended), each enclosed in curly braces.
% Keywords are useful for indexing and searching for the theses by topic.
\def\Keywords{%
\xxx{{key} {words}}
}

% If your abstracts are long and do not fit in the infopage, you can make the
% fonts a bit smaller by this setting. (Also, you should try to compress your abstract more.)
% Alternatively, consider increasing the size of the page by uncommenting the
% geometry modification in thesis.tex.
\def\InfoPageFont{}
%\def\InfoPageFont{\small}  %uncomment to decrease font size

\ifEN\relax\else
% If you are writing a czech thesis, you additionally need to fill in the
% english translation of the metadata here!
\def\ThesisTitleEN{\xxx{Thesis title in English}}
\def\DepartmentEN{\xxx{Name of the department in English}}
\def\DeptTypeEN{\xxx{Department}}
\def\SupervisorsDepartmentEN{\xxx{Superdepartment}}
\def\StudyProgrammeEN{\xxx{study programme}}
\def\StudyBranchEN{\xxx{study branch}}
\def\KeywordsEN{%
\xxx{{key} {words}}
}
\fi


\usepackage[a-2u]{pdfx}

\ifEN\else\usepackage[czech,shorthands=off]{babel}\fi
\usepackage[utf8]{inputenc}
\usepackage[T1]{fontenc}

% See https://en.wikipedia.org/wiki/Canons_of_page_construction before
% modifying the size of printable area. LaTeX defaults are great.
% If you feel it would help anything, you can enlarge the printable area a bit:
%\usepackage[textwidth=390pt,textheight=630pt]{geometry}
% The official recommendation expands the area quite a bit (looks pretty harsh):
%\usepackage[textwidth=145mm,textheight=247mm]{geometry}

%%% FONTS %%%
\usepackage{lmodern} % TeX "original" (this sets up the latin mono)

% Optionally choose an override for the main font for typesetting:
\usepackage[mono=false]{libertinus} % popular for comp-sci (ACM uses this)
%\usepackage{tgschola} % Schoolbook-like (gives a bit of historic feel)
%\usepackage[scale=0.96]{tgpagella} % Palladio-like (popular in formal logic).
% IBM Plex font suite is nice but requires us to fine-tune the sizes, also note
% that it does not directly support small caps (\textsc) and requires lualatex:
%\usepackage[usefilenames,RM={Scale=0.88},SS={Scale=0.88},SScon={Scale=0.88},TT={Scale=0.88},DefaultFeatures={Ligatures=Common}]{plex-otf}

% Optionally, choose a custom sans-serif fonts (e.g. for figures and tables).
% Default sans-serif font is usually Latin Modern Sans. Some font packages
% (e.g. libertinus) replace that with a better matching sans-serif font.
%\usepackage{tgheros} % recommended and very readable (Helvetica-like)
%\usepackage{FiraSans} % looks great
% DO NOT typeset the main text in sans-serif font!
% The serifs make the text easily readable on the paper.


% IMPORTANT FONT NOTE: Some fonts require additional PDF/A conversion using
% the pdfa.sh script. These currently include only 'tgpagella'; but various
% other fonts from the texlive distribution need that too (mainly the Droid
% font family).


% some useful packages
\usepackage{microtype}
\usepackage{amsmath,amsfonts,amsthm,bm}
\usepackage{graphicx}
\usepackage{xcolor}
\usepackage{booktabs}
\usepackage{caption}
\usepackage{floatrow}
\usepackage{tikz}
\usepackage{hyperref}

% load bibliography tools
\usepackage[backend=bibtex,natbib,style=numeric,sorting=none]{biblatex}
% alternative with alphanumeric citations (more informative than numbers):
%\usepackage[backend=bibtex,natbib,style=alphabetic]{biblatex}
%
% alternatives that conform to iso690
% (iso690 is not formally required on MFF, but may help elsewhere):
%\usepackage[backend=bibtex,natbib,style=iso-numeric,sorting=none]{biblatex}
%\usepackage[backend=bibtex,natbib,style=iso-alphabetic]{biblatex}
%
% additional option choices:
%  - add `giveninits=true` to typeset "E. A. Poe" instead of full Edgar Allan
%  - `terseinits=true` additionaly shortens it to nature-like "Poe EA"
%  - add `maxnames=10` to limit (or loosen) the maximum number of authors in
%    bibliography entry before shortening to `et al.` (useful when referring to
%    book collections that may have hundreds of authors)
%  - for additional flexibility (e.g. multiple reference sections, etc.),
%    remove `backend=bibtex` and compile with `biber` instead of `bibtex` (see
%    Makefile)
%  - `sorting=none` causes the bibliography list to be ordered by the order of
%    citation as they appear in the text, which is usually the desired behavior
%    with numeric citations. Additionally you can use a style like
%    `numeric-comp` that compresses the long lists of citations such as
%    [1,2,3,4,5,6,7,8] to simpler [1--8]. This is especially useful if you plan
%    to add tremendous amounts of citations, as usual in life sciences and
%    bioinformatics.
%  - if you don't like the "In:" appearing in the bibliography, use the
%    extended style (`ext-numeric` or `ext-alphabetic`), and add option
%    `articlein=false`.
%
% possibly reverse the names of the authors with the default styles:
%\DeclareNameAlias{default}{family-given}

% load the file with bibliography entries
\addbibresource{refs}

% remove this if you won't use fancy verbatim environments
\usepackage{fancyvrb}

% remove this if you won't typeset TikZ graphics
\usepackage{tikz}
\usetikzlibrary{positioning} %add libraries as needed (shapes, decorations, ...)

% remove this if you won't typeset any pseudocode
\usepackage{algpseudocode}
\usepackage{algorithm}

% remove this if you won't list any source code
\usepackage{listings}


\hypersetup{unicode}
\hypersetup{breaklinks=true}

\usepackage[noabbrev]{cleveref}


% various forms of TODOs (you should remove this before submitting)
\usepackage[textsize=tiny, backgroundcolor=yellow!25, linecolor=black!25]{todonotes}
\newcommand{\xxx}[1]{\textcolor{red!}{#1}}

 % remove this before compiling the final version


% use this for typesetting a chapter without a number, e.g. intro and outro
\def\chapwithtoc#1{\chapter*{#1}\addcontentsline{toc}{chapter}{#1}}

% If there is a line/figure overflowing into page margin, this will make the
% problem evident by drawing a thick black line at the overflowing spot. You
% should not disable this.
\overfullrule=3mm

% The maximum stretching of a space. Increasing this makes the text a bit more
% sloppy, but may prevent the overflows by moving words to next line.
\emergencystretch=1em

\ifEN
\theoremstyle{plain}
\newtheorem{thm}{Theorem}
\newtheorem{lemma}[thm]{Lemma}
\newtheorem{claim}[thm]{Claim}
\newtheorem{defn}{Definition}
\theoremstyle{remark}
\newtheorem*{cor}{Corollary}
\else
\theoremstyle{plain}
\newtheorem{thm}{Věta}
\newtheorem{lemma}{Lemma}
\newtheorem{claim}{Tvrzení}
\newtheorem{defn}{Definice}
\theoremstyle{remark}
\newtheorem*{cor}{Důsledek}
\fi

\newenvironment{myproof}{
  \par\medskip\noindent
  \textit{\ifEN Proof \else Důkaz \fi}.
}{
\newline
\rightline{$\qedsymbol$}
}

% real/natural numbers
\newcommand{\R}{\mathbb{R}}
\newcommand{\N}{\mathbb{N}}

% asymptotic complexity
\newcommand{\asy}[1]{\mathcal{O}(#1)}

% listings and default lstlisting config (remove if unused)
\DeclareNewFloatType{listing}{}
\floatsetup[listing]{style=ruled}

\DeclareCaptionStyle{thesis}{style=base,font={small,sf},labelfont=bf,labelsep=quad}
\captionsetup{style=thesis}
\captionsetup[algorithm]{style=thesis,singlelinecheck=off}
\captionsetup[listing]{style=thesis,singlelinecheck=off}

% Customization of algorithmic environment (comment style)
\renewcommand{\algorithmiccomment}[1]{\textcolor{black!25}{\dotfill\sffamily\itshape#1}}

% Uncomment for table captions on top. This is sometimes recommended by the
% style guide, and even required for some publication types.
%\floatsetup[table]{capposition=top}
%
% (Opinionated rant:) Captions on top are not "compatible" with the general
% guideline that the tables should be formatted to be quickly visually
% comprehensible and *beautiful* in general (like figures), and that the table
% "head" row (with column names) should alone communicate most of the content
% and interpretation of the table. If you just need to show a long boring list
% of numbers (because you have to), either put some effort into showing the
% data in an attractive figure-table, or move the data to an attachment and
% refer to it, so that the boredom does not impact the main text flow.
%
% You can make the top-captions look much less ugly by aligning the widths of
% the caption and the table, with setting `framefit=yes`, as shown below.  This
% additionally requires some extra markup in your {table} environments; see the
% comments in the example table in `ch2.tex` for details.
%\floatsetup[table]{capposition=top,framefit=yes}

\ifEN\floatname{listing}{Listing}
\else\floatname{listing}{Výpis kódu}\fi
\lstset{ % use this to define styling for any other language
  language=C++,
  tabsize=2,
  showstringspaces=false,
  basicstyle=\footnotesize\tt\color{black!75},
  identifierstyle=\bfseries\color{black},
  commentstyle=\color{green!50!black},
  stringstyle=\color{red!50!black},
  keywordstyle=\color{blue!75!black}}

% Czech versions of the used cleveref references (It's not as convenient as in
% English because of declension, cleveref is limited to sg/pl nominative. Use
% plain \ref to dodge that.)
\ifEN\relax\else
\crefname{chapter}{kapitola}{kapitoly}
\Crefname{chapter}{Kapitola}{Kapitoly}
\crefname{section}{sekce}{sekce}
\Crefname{section}{Sekce}{Sekce}
\crefname{subsection}{sekce}{sekce}
\Crefname{subsection}{Sekce}{Sekce}
\crefname{subsubsection}{sekce}{sekce}
\Crefname{subsubsection}{Sekce}{Sekce}
\crefname{figure}{obrázek}{obrázky}
\Crefname{figure}{Obrázek}{Obrázky}
\crefname{table}{tabulka}{tabulky}
\Crefname{table}{Tabulka}{Tabulky}
\crefname{listing}{výpis}{výpisy}
\Crefname{listing}{Výpis}{Výpisy}
\floatname{algorithm}{Algoritmus}
\crefname{algorithm}{algoritmus}{algoritmy}
\Crefname{algorithm}{Algoritmus}{Algoritmy}
\newcommand{\crefpairconjunction}{ a~}
\newcommand{\crefrangeconjunction}{ a~}
\fi
 % use this file for various custom definitions


\begin{document}

% the layout is mandatory, edit only in dire circumstances

\pagestyle{empty}
\hypersetup{pageanchor=false}
\begin{center}

% top part of the layout, this actually differs between faculties

\def\ThesisTitleXmff{%
  \ifEN
    \centerline{\mbox{\includegraphics[width=166mm]{img/logo-en.pdf}}}
  \else
    \centerline{\mbox{\includegraphics[width=166mm]{img/logo-cs.pdf}}}
  \fi
  \vspace{-8mm}\vfill%
  {\bf\Large\ThesisTypeTitle}
  \vfill%
  {\LARGE\ThesisAuthor}\par
  \vspace{15mm}%
  {\LARGE\bfseries\ThesisTitle}
  \vfill%
  \Department}
\def\ThesisTitleCuniLogo#1{%
  {\large\UKName\par\medskip\par\UKFaculty }
  \vfill%
  {\bf\Large\ThesisTypeTitle}
  \vfill%
  \includegraphics[width=70mm]{#1}
  \vfill%
  {\LARGE\ThesisAuthor}\par
  \vspace{15mm}%
  {\LARGE\bfseries\ThesisTitle}
  \vfill%
  \Department\par}
\def\ThesisTitleXcuni{\ThesisTitleCuniLogo{img/uklogo.pdf}}
\def\ThesisTitleXnatur{\ThesisTitleCuniLogo{img/naturlogo.pdf}}

% choose the correct page and print it
\csname ThesisTitleX\ThesisTitleStyle\endcsname
% latex corner: X is the new @

\vfill

{
\centerline{\vbox{\halign{\hbox to 0.45\hsize{\hfil #}&\hskip 0.5em\parbox[t]{0.45\hsize}{\raggedright #}\cr
\ifEN Supervisor of the \ThesisGenitive thesis:
\else Vedoucí \ThesisGenitive práce: \fi
& \Supervisor \cr
\noalign{\vspace{2mm}}
\ifEN Study programme: \else Studijní program: \fi
& \StudyProgramme \cr
\noalign{\vspace{2mm}}
\ifEN Study branch: \else Studijní obor: \fi
& \StudyBranch \cr
}}}}

\vfill

\ifEN Prague \else Praha \fi
\YearSubmitted

\end{center}

\newpage

% remember to sign this!
\openright
\hypersetup{pageanchor=true}
\pagestyle{plain}
\pagenumbering{roman}
\vglue 0pt plus 1fill

\ifEN
\noindent
I declare that I carried out this \ThesisAccusative thesis independently, and only with the cited
sources, literature and other professional sources. It has not been used to obtain another
or the same degree.
\else
\noindent
Prohlašuji, že jsem tuto \ThesisAccusative práci vypracoval(a) samostatně a výhradně
s~použitím citovaných pramenů, literatury a dalších odborných zdrojů.
Tato práce nebyla využita k získání jiného nebo stejného titulu.
\fi

\ifEN
\medskip\noindent
I understand that my work relates to the rights and obligations under the Act No.~121/2000 Sb.,
the Copyright Act, as amended, in particular the fact that the Charles
University has the right to conclude a license agreement on the use of this
work as a school work pursuant to Section 60 subsection 1 of the Copyright~Act.
\else
\medskip\noindent
Beru na~vědomí, že se na moji práci vztahují práva a povinnosti vyplývající
ze zákona č. 121/2000 Sb., autorského zákona v~platném znění, zejména skutečnost,
že Univerzita Karlova má právo na~uzavření licenční smlouvy o~užití této
práce jako školního díla podle §60 odst. 1 autorského zákona.
\fi

\vspace{10mm}


\ifEN
\hbox{\hbox to 0.5\hsize{%
In \hbox to 6em{\dotfill} date \hbox to 6em{\dotfill}
\hss}\hbox to 0.5\hsize{\dotfill\quad}}
\smallskip
\hbox{\hbox to 0.5\hsize{}\hbox to 0.5\hsize{\hfil Author's signature\hfil}}
\else
\hbox{\hbox to 0.5\hsize{%
V \hbox to 6em{\dotfill} dne \hbox to 6em{\dotfill}
\hss}\hbox to 0.5\hsize{\dotfill\quad}}
\smallskip
\hbox{\hbox to 0.5\hsize{}\hbox to 0.5\hsize{\hfil Podpis autora\hfil}}
\fi

\vspace{20mm}
\newpage

% dedication

\openright

\noindent
\Dedication

\newpage

% mandatory information page

\openright

\vbox to 0.49\vsize{\InfoPageFont
\setlength\parindent{0mm}
\setlength\parskip{5mm}

\ifEN Title: \else Název práce: \fi
\ThesisTitle

\ifEN Author: \else Autor: \fi
\ThesisAuthor

\DeptType:
\Department

\ifEN Supervisor: \else Vedoucí bakalářské práce: \fi
\Supervisor, \SupervisorsDepartment

\ifEN Abstract: \AbstractEN \else Abstrakt: \AbstractCS \fi

\ifEN Keywords: \else Klíčová slova: \fi
\Keywords

\vss}\ifEN\relax\else\nobreak\vbox to 0.49\vsize{\InfoPageFont
\setlength\parindent{0mm}
\setlength\parskip{5mm}

Title:
\ThesisTitleEN

Author:
\ThesisAuthor

\DeptTypeEN:
\DepartmentEN

Supervisor:
\Supervisor, \SupervisorsDepartmentEN

Abstract:
\AbstractEN

Keywords:
\KeywordsEN

\vss}
\fi

\newpage

\openright
\pagestyle{plain}
\pagenumbering{arabic}
\setcounter{page}{1}


\tableofcontents

\chapwithtoc{Introduction}

\section*{What are skyrmions?}

\citet{chen2020nanoscale}

Skyrmion originates from particle theory, describing topologically stable field configurations with particle-like properties. Tony Skyrme introduced it as a potential model of the nucleon in 1961. Although this model turned out incorrect, skyrmions, as topological objects, also re-emerged in various other fields. 

This thesis deals with \emph{magnetic skyrmions}, first experimentally observed in 2009 \cite{mühlbauer2009skyrmion} in chiral magnets. They are 2-D vertex-like structures characterized by rotationally symmetric, continuously changing spin vectors from their middle toward their edges. Besides chiral magnets, magnetic skyrmions were observed in ultra-thin films and multilayer structures. The non-trivial topology of skyrmions makes them quite robust, meaning they can withstand small perturbations caused, for example, by thermal fluctuations or material defects. Stability, with their small size (nanometer scale) and very energy-efficient manipulation, makes them an interesting object for application, particularly in spintronics. They can be used as bits in next-generation magnetic memory devices with much higher density and lower energy requirements.

The formation of magnetic skyrmions is typically conditional on two external fields: The first and most important is the so-called Dzyaloshinskii-Moriya (DM) interaction, which prefers an anti-parallel spin configuration. The second is the magnetic field (B), which is, in some cases, necessary to isolate and stabilize individual skyrmions.  

\section*{Problem statement}

Assuming a constant temperature, low enough to allow skyrmions to form, their size depends on the levels of the external fields. Also, with changing external fields, skyrmions undergo a continuous phase transition towards a ferromagnet or a degenerate skyrmion phase called a magnetic spiral. This nature makes it difficult to define strict phase boundaries, which is why a statistical approach can be beneficial. Our idea is based on two steps: First step is Monte Carlo simulation of spin configuration at different levels oiakovlev2018supervisedf external fields. We generate a number of configurations for all 3 phases, also capturing the continuous phase transitions. We take a projection to the $z$-axis for each spin in the configuration and treat this projection as image. The second step is a deployment a statistical model, where we use the images as its inputs. Later step counts on the proven ability of machine-learning (ML) models to recognize complex patterns in high-dimensional data. With this approach, we aim to achieve more precise classification, which would result in a deeper understanding of the phase diagram, providing useful insight for the general study of magnetic systems with topological structures.

\section*{Approach to solve the problem}

Multiple machine-learning approaches has already been applied to this problem in \citet{iakovlev2018supervised}, where is was proved that simple dense neural network is able to classify these 3-phases with reasonable accuracy. However, field of ML provides much richer options then were considered in the mentioned paper. We will build, and try to improve on these results by experimenting with various architectures and systematically exploring numerous options one has when defining and training a statistical model. We devote special attention to evaluating various regularization techniques such as data augmentation or dropout Monte Carlo.

ML models are often looked upon as black boxes, which is often truth especially in result driven field. Therefore we prefer models with good interpretation and explore our options systematically. This way, we can gain a general insight into how the model behaves and why, and with such an insight we are then able to make rational choices in model architecture and have more power in interpretation of the results. Therefore, this thesis will also be a very useful resource for the application of ML models in this problem for real experimental data, obtained e.g. by Lorenz transmission electron microscopy or spin-polarized scanning tunneling microscopy.

\section*{Overview of next chapters}

In \autoref{chap1} we will introduce magnetic skyrmions in more depth focusing on the physical background. We will continue by detailed explanation of simulation of our dataset, and explore its difficulties.


In \autoref{chap2} we will explain important machine learning concepts and define model architectures used in our numerical experiments. Then we bring special attention to different kinds of regularization techniques used in our models.


In \autoref{chap3} we present the results comparing different model architectures and different regularization techniques.


\chapter{Skyrmion}
\label{chap1}

\section{Magnetic skyrmions}

Magnetic skyrmions are 2-D topologically stable spin formations

The essential role in the formation of magnetic skyrmions plays Dzyaloshinskii–Moriya (DM) interaction (DMI), also called antisymetric exchange. 

\xxx{Wiki: The discovery of antisymmetric exchange originated in the early 20th century from the controversial observation of weak ferromagnetism in typically antiferromagnetic Fe2O3 crystals.[1] In 1958, Igor Dzyaloshinskii provided evidence that the interaction was due to the relativistic spin lattice and magnetic dipole interactions based on Lev Landau's theory of phase transitions of the second kind.[2] In 1960, Toru Moriya identified the spin-orbit coupling as the microscopic mechanism of the antisymmetric exchange interaction.[1] Moriya referred to this phenomenon specifically as the "antisymmetric part of the anisotropic superexchange interaction." The simplified naming of this phenomenon occurred in 1962, when D. Treves and S. Alexander of Bell Telephone Laboratories simply referred to the interaction as antisymmetric exchange. Because of their seminal contributions to the field, antisymmetric exchange is sometimes referred to as the Dzyaloshinskii–Moriya interaction.}

It is a magnetic exchange interaction between neighbouring patricles with spin. It can be charecterized by hamiltonian:

\begin{equation}
	\label{DM-interaction-hamiltonian}
	H^{\text{DM}}_{ij} = \bm{D}_{ij} \dot (\bm{S}_i \times \bm{S}_j),
\end{equation}

where $\bm{D}_{ij}$ is a Dzyaloshinskii–Moriya vector characterizing the interaction, and $\bm{S}_i$, $\bm{S}_j$ are spin vectors. 

This means that the DM interaction prefers perpendicular orientation of neighboring spins. The vector $\bm{D}_{ij}$ is given by the symmetry of the system. DMI mediates also one other stable magnetic arrangement which is a \emph{spin spiral}. 

The direction of vector $\bm{D}_{ij}$ is dictated by the system symmetry, and influences the topology of skyrmions and spirals.

Main source \cite{li2023magnetic}



\section{Dataset}

\todo{how do we get our data}

Defining part of any machine learning problem is the dataset we are working with. Our dataset consists of in total 9840 images of magnetic configurations created by Monte-Carlo simulations on discrete 200x200 square grid with periodic boundry conditions near absolute zero temperature. The simulations were done in conditions given be values of $D$ and $B$. Used values of $D$ ranged from We have 5 simulated different configurations for each pair of $D$ and $B$ values.


\todo{chalenges of this dataset}

\todo{train dataset}

\todo{one-hot encoding for labels}



\chapter{Regularization}

\todo{consider exploring double decent phenomenon in our usecase (beware of possible counteraction with augmentation)}

The main source for this chapter is \cite{bengio2017deep}

The key concept for this thesis is the term generalization. Generalization is the ability of the model to perform well on previosly unseen data. In our case we train the model on the clean single phase images and want the model to perform well also on the images near the phase transitions where there are often simultaniosly two phases present in the different parts of the given image. 

There are many strategies how to improve generalization (often with the cost of increasing training error). Some of these strategies relies on putting constrains on the model parameters, some modify the objective function. Some strategies modify the training data, some interfere with the training process, model's architecture and many other things. We will describe strategies that are relevant for us in the following section.

\subsection{Parameter-norm penalties}
A common way to regularize, used long before deeplearning, is to limit the capacity of a given model. This can be achieved by adding parameter-norm $\Omega(\bm{\theta})$ to the loss function $L(\bm{x}; \bm{\theta})$

\begin{equation}
	\label{eq:loss-parameter-norm}
	\tilde{L(\bm{x}; \bm{\theta})} = L(\bm{x}; \bm{\theta}) + \alpha \Omega(\bm{\theta}),
\end{equation}
where $\alpha \geq 0$ is a hyperparameter weighting the contribution of the parameter-norm to the overall loss function.

This change means that will try to minize both the original loss function and some norm of the models parameters during the training. Noticeably $\Omega$ doesn't have to penalize all the parameters $\bm{\theta}$, in fact it is recommended to reagularize only proper weights $\bm{w}$ (leaving the biases $\bm{b}$ out) in this manner. While proper weights try to capture capture interactions between pairs of variables, biases only shifts the overall output. Large biases do not increase variability in activations and on the other hand having a large bias might be sometimes preferable. Therefore penalizing the biases in this way can cause significant and unnecessary underfitting.


The most common parameter norms are $L^2$:
\begin{equation}
 	\label{eq:l2-regularization}
 	\Omega(\bm{\theta}) = \dfrac{1}{2} ||\bm{w}||_2^2,
\end{equation} a
nd $L^1$:
\begin{equation}
	\label{eq:l1-regularization}
	\Omega(\bm{\theta}) = \dfrac{1}{2} ||\bm{w}||_1.
\end{equation} 


\subsection{Weight decay}
A weight decay is very similar to $L^2$ regularization, but is implemented little differently. Instead of changing the loss function, it modifies the weight update rule during minimization. 


......

\subsection{Data augmentation}

The best way to improve generalization of machine learning model, is to train it on larger dataset. Getting new data is not always practical, but we can help ourselves by modifying data we already have. For our advantage, this approach is easy to apply in image classification tasks. We can rotate by 90 degrees, reflext and due to periodical boundary conditions also translate the images. These tranformations in no way change the information, and they add additional variability to our data, making it effectivelly larger. Yet this is not the biggest leverage data augmentation brings for our usecase. We can also combine images from different classes in order to help our model recognize phase transitions. There are two common approaches for such data augmentation, CutMix and MixUp. CutMix works by taking a portion of image 1 (usually rectangular segment) and pasting it onto an image 2. Resulting image will get label that is a weighted sum of original one-hot encoded labels, with weights proportional to number of pixels contributed by each image. MixUp works again by taking two images, generating random number $\alpha \in (0, 1)$ and adding pixels of these images weighted by $\alpha$ (resp. $1 - \alpha$). Label is then created in same manner.


\chapter{More complicated chapter}
\label{chap:math}

After the reader gained sufficient knowledge to understand your problem in \cref{chap:refs}, you can jump to your own advanced material and conclusions.

You will need definitions (see \cref{defn:x} below in \cref{sec:demo}), theorems (\cref{thm:y}), general mathematics, algorithms (\cref{alg:w}), and tables (\cref{tab:z})\todo{See documentation of package \texttt{booktabs} for hints on typesetting tables. As a main rule, \emph{never} draw a vertical line.}. \Cref{fig:f,fig:g} show how to make a nice figure. See \cref{fig:schema} for an example of TikZ-based diagram. Cross-referencing helps to keep the necessary parts of the narrative close --- use references to the previous chapter with theory wherever it seems that the reader could have forgotten the required context. Conversely, it is useful to add a few references to theoretical chapters that point to the sections which use the developed theory, giving the reader easy access to motivating application examples.

\section{Example with some mathematics}
\label{sec:demo}

\begin{defn}[Triplet]\label{defn:x}
Given stuff $X$, $Y$ and $Z$, we will write a \emph{triplet} of the stuff as $(X,Y,Z)$.
\end{defn}

\newcommand{\Col}{\textsc{Colour}}

\begin{thm}[Car coloring]\label{thm:y}
All cars have the same color. More specifically, for any set of cars $C$, we have
$$(\forall c_1, c_2 \in C)\:\Col(c_1) = \Col(c_2).$$
\end{thm}

\begin{proof}
Use induction on sets of cars $C$. The statement holds trivially for $|C|\leq1$. For larger $C$, select 2 overlapping subsets of $C$ smaller than $|C|$ (thus same-colored). Overlapping cars need to have the same color as the cars outside the overlap, thus also the whole $C$ is same-colored.\todo{This is plain wrong though.}
\end{proof}

\begin{table}
% uncomment the following line if you use the fitted top captions for tables
% (see the \floatsetup[table] comments in `macros.tex`.
%\floatbox{table}[\FBwidth]{
\centering\footnotesize\sf
\begin{tabular}{llrl}
\toprule
Column A & Column 2 & Numbers & More \\
\midrule
Asd & QWERTY & 123123 & -- \\
Asd qsd 1sd & \textcolor{red}{BAD} & 234234234 & This line should be helpful. \\
Asd & \textcolor{blue}{INTERESTING} & 123123123 & -- \\
Asd qsd 1sd & \textcolor{violet!50}{PLAIN WEIRD} & 234234234 & -- \\
Asd & QWERTY & 123123 & -- \\
\addlinespace % a nice non-intrusive separator of data groups (or final table sums)
Asd qsd 1sd & \textcolor{green!80!black}{GOOD} & 234234299 & -- \\
Asd & NUMBER & \textbf{123123} & -- \\
Asd qsd 1sd & \textcolor{orange}{DANGEROUS} & 234234234 & (no data) \\
\bottomrule
\end{tabular}
%}{  % uncomment if you use the \floatbox (as above), erase otherwise
\caption{An example table.  Table caption should clearly explain how to interpret the data in the table. Use some visual guide, such as boldface or color coding, to highlight the most important results (e.g., comparison winners).}
%}  % uncomment if you use the \floatbox
\label{tab:z}
\end{table}

\begin{figure}
\centering
\includegraphics[width=.6\linewidth]{img/ukazka-obr02.pdf}
\caption{A figure with a plot, not entirely related to anything. If you copy the figures from anywhere, always refer to the original author, ideally by citation (if possible). In particular, this picture --- and many others, also a lot of surrounding code --- was taken from the example bachelor thesis of MFF, originally created by Martin Mareš and others.}
\label{fig:g}
\end{figure}

\begin{figure}
\centering
\tikzstyle{box}=[rectangle,draw,rounded corners=0.5ex,fill=green!10]
\begin{tikzpicture}[thick,font=\sf\scriptsize]
\node[box,rotate=45] (a) {A test.};
\node[] (b) at (4,0) {Node with no border!};
\node[circle,draw,dashed,fill=yellow!20, text width=6em, align=center] (c) at (0,4) {Ugly yellow node.\\Is this the Sun?};
\node[box, right=1cm of c] (d) {Math: $X=\sqrt{\frac{y}{z}}$};
\draw[->](a) to (b);
\draw[->](a) to[bend left=30] node[midway,sloped,anchor=north] {flow flows} (c);
\draw[->>>,dotted](b) to[bend right=30] (d);
\draw[ultra thick](c) to (d);

\end{tikzpicture}
\caption{An example diagram typeset with TikZ. It is a good idea to write diagram captions in a way that guides the reader through the diagram. Explicitly name the object where the diagram viewing should ``start''. Preferably, briefly summarize the connection to the parts of the text and other diagrams or figures. (In this case, would the tenative yellow Sun be described closer in some section of the thesis? Or, would there be a figure to detail the dotted pattern of the line?)}
\label{fig:schema}
\end{figure}

\begin{algorithm}
\begin{algorithmic}
\Function{ExecuteWithHighProbability}{$A$}
	\State $r \gets$ a random number between $0$ and $1$
	\State $\varepsilon \gets 0.0000000000000000000000000000000000000042$
	\If{$r\geq\varepsilon$}
		\State execute $A$ \Comment{We discard the return value}
	\Else
		\State print: \texttt{Not today, sorry.}
	\EndIf
\EndFunction
\end{algorithmic}
\caption{Algorithm that executes an action with high probability. Do not care about formal semantics in the pseudocode --- semicolons, types, correct function call parameters and similar nonsense from `realistic' languages can be safely omitted. Instead make sure that the intuition behind (and perhaps some hints about its correctness or various corner cases) can be seen as easily as possible.}
\label{alg:w}
\end{algorithm}

\section{Extra typesetting hints}

Do not overuse text formatting for highlighting various important parts of your sentences. If an idea cannot be communicated without formatting, the sentence probably needs rewriting anyway. Imagine the thesis being read aloud as a podcast --- the storytellers are generally unable to speak in boldface font.

Most importantly, do \underline{not} overuse bold text, which is designed to literally \textbf{shine from the page} to be the first thing that catches the eye of the reader. More precisely, use bold text only for `navigation' elements that need to be seen and located first, such as headings, list item leads, and figure numbers.

Use underline only in dire necessity, such as in the previous paragraph where it was inevitable to ensure that the reader remembers to never typeset boldface text manually again.

Use \emph{emphasis} to highlight the first occurrences of important terms that the reader should notice. The feeling the emphasis produces is, roughly, ``Oh my --- what a nicely slanted word! Surely I expect it be important for the rest of the thesis!''

Finally, never draw a vertical line, not even in a table or around figures, ever. Vertical lines outside of the figures are ugly.

\chapter{Results and discussion}

You should have a separate chapter for presenting your results (generated by the stuff described previously, in our case in \cref{chap:math}). Remember that your work needs to be validated rigorously, and no one will believe you if you just say that `it worked well for you'.

Instead, try some of the following:
\begin{itemize}
\item State a hypothesis and prove it statistically
\item Show plots with measurements that you did to prove your results (e.g. speedup). Use either \texttt{R} and \texttt{ggplot}, or Python with \texttt{matplotlib} to generate the plots.\footnote{Honestly, the plots from \texttt{ggplot} look \underline{much} better.} Save them as PDF to avoid printing pixels (as in \cref{fig:f}).
\item Compare with other similar software/theses/authors/results, if possible
\item Show example source code (e.g. for demonstrating how easily your results can be used)
\item Include a `toy problem' for demonstrating the basic functionality of your approach and detail all important properties and results on that
\item Include clear pictures of `inputs' and `outputs' of all your algorithms, if applicable
\end{itemize}

\begin{figure}
\centering
\includegraphics[width=.6\linewidth]{img/ukazka-obr01.pdf}
\caption{This caption is a friendly reminder to never insert figures ``in text,'' without a floating environment, unless explicitly needed for maintaining the text flow (e.g., the figure is small and developing with the text, like some of the centered equations, as in \cref{thm:y}). All figures \emph{must} be referenced by number from the text (so that the readers can find them when they read the text) and properly captioned (so that the readers can interpret the figure even if they look at it before reading the text --- reviewers love to do that).}
\label{fig:f}
\end{figure}

It is sometimes convenient (even recommended by some journals, including Cell) to name the results sub-sections so that they state what exactly has been achieved. Examples follow.

\section{SuperProgram is faster than OldAlgorithm}
\subsection{Scalability estimation}
\subsection{Precision of the results}
\section{Weird theorem is proven by induction}
\section{Amount of code reduced by CodeRedTool}
\subsection{Example}
\subsection{Performance on real codebases}
\section{\sloppy NeuroticHelper improves neural network learning}

\section{Graphics and figure quality}

No matter how great the text content of your thesis is, the pictures will always catch the attention first. This creates the very important first impression of the thesis contents and general quality. Crucially, that also decides whether the thesis is later read with joy, or carefully examined with suspicion.

Preparing your thesis in a way such that this first impression gets communicated smoothly and precisely helps both the reviewer and you: the reviewer will not have a hard time understanding what exactly you wanted to convey, and you will get a better grade.

Making the graphics `work for you' involves doing some extra work that is often unexpected. At the same time, you will need to fit into graphics quality constraints and guidelines that are rarely understood before you actually see a bad example. As a rule of thumb, you should allocate at least the same amount of time and effort for making the figures look good as you would for writing, editing and correcting the same page area of paragraph text.

\subsection{Visualize all important ideas}
The set of figures in your thesis should be comprehensive and complete. For all important ideas, constructions, complicated setups and results there should be a visualization that the reader can refer to in case the text does not paint the `mental image' sufficiently well. At the bare minimum, you should have at least 3 figures (roughly corresponding to the 3 chapters) that clearly and unambiguously show:
\begin{enumerate}
\item the context of the problem you are solving, optionally with e.g.~question marks and exclamation marks placed to highlight the problems and research questions
\item the overall architecture of your solution (usually as a diagram with arrows, such as in \cref{fig:schema}, ideally with tiny toy examples of the inputs and outputs of each box),
\item the advancement or the distinctive property of your solution, usually in a benchmark plot, or as a clear demonstration and comparison of your results.
\end{enumerate}

\subsection{Make the figures comprehensible}
The figures should be easily comprehensible. Surprisingly, that requires you to follow some common ``standards'' in figure design and processing. People are often used to a certain form of the visualizations, and (unless you have a very good reason) deviating from the standard is going to make the comprehension much more complicated. The common standards include the following:
\begin{itemize}
  \item caption everything correctly, place the caption at an expectable position
  \item systematically label the plots with `main' titles (usually in boldface, above the plot), plot axes, axis units and ticks, and legends
  \item lay out the diagrams systematically, ideally follow a structure of a bottom-up tree, a left-to-right pipeline, a top-down layered architecture, or a center-to-borders mindmap
  \item {use colors that convey the required information correctly \par\footnotesize Although many people carry some intuition for color use, achieving a really correct utilization of colors is often very hard without previous experience in color science and typesetting. Always remember that everyone perceives color hues differently, therefore the best distinction between the colors is done by varying lightness of the graphics elements (i.e., separating the data by dark vs.~light) rather than by using hues (i.e., forcing people to guess which one of salmon and olive colors means ``better''). Almost 10\% of the population have their vision impaired by some form of color vision deficiency, most frequently by deuteranomaly that prevents interpretation of even the most `obvious' hue differences, such as green vs.~red. Finally, printed colors look surprisingly different from the on-screen colors. You can prevent much of these problems by using standardized palettes and well-tested color gradients, such as the ones from ColorBrewer\footnote{\url{https://colorbrewer2.org}} and ViridisLite\footnote{\url{https://sjmgarnier.github.io/viridisLite/}}. Check if your pictures still look good if converted to greyscale, and use a color deficiency simulator to check how the colors are perceived with deuteranomaly.}
\end{itemize}

Avoid large areas of over-saturated and dark colors:
\begin{itemize}
  \item under no circumstances use dark backgrounds for any graphical elements, such as diagram boxes and tables --- use very light, slightly desaturated colors instead
  \item avoid using figures that contain lots of dark color (as a common example, heatmaps rendered with the `magma' color palette often look like huge black slabs that are visible even through the paper sheet, thus making a dark smudge on the neighboring page)
  \item increase the brightness of any photos to match the average brightness of the text around the figure
\end{itemize}

Remember to test your figures on other people --- usually, just asking `What do you think the figure should show?' can help you debug many mistakes in your graphics. If they think that the figure says something different than what you planned, then most likely it is your figure what is wrong, not the understanding of others.

Finally, there are many magnificent resources that help you arrange your graphics correctly. The two books by Tufte~\cite{tufte1990envisioning,tufte1983visual} are arguably classics in the area. Additionally, you may find many interesting resources to help you with technical aspects of plotting, such as the \texttt{ggplot}-style `Fundamentals' book by~\citet{wilke2019fundamentals}, and a wonderful manual for the TikZ/PGF graphics system by~\citet{tantau2015tikz} that will help you draw high-quality diagrams (like the one in~\cref{fig:schema}).

\section{What is a discussion?}
After you present the results and show that your contributions work, it is important to \emph{interpret} them, showing what they mean in the wider context of the thesis topic, for the researchers who work in the area, and for the more general public, such as for the users.

Separate discussion sections are therefore common in life sciences where some ambiguity in result interpretation is common, and the carefully developed intuition about the wider context is sometimes the only thing that the authors have. Exact sciences and mathematicians do not need to use the discussion sections as often. Despite of that, it is nice to position your output into the previously existing environment, answering:
\begin{itemize}
\item What is the potential application of the result?
\item Does the result solve a problem that other people encountered?
\item Did the results point to any new (surprising) facts?
\item How (and why) is the approach you chose different from what the others have done previously?
\item Why is the result important for your future work (or work of anyone other)?
\item Can the results be used to replace (and improve) anything that is used currently?
\end{itemize}

If you do not know the answers, you may want to ask the supervisor. Also, do not worry if the discussion section is half-empty or completely pointless; you may remove it completely without much consequence. It is just a bachelor thesis, not a world-saving avenger thesis.


\chapwithtoc{Conclusion}

In the conclusion, you should summarize what was achieved by the thesis. In a few paragraphs, try to answer the following:
\begin{itemize}
\item Was the problem stated in the introduction solved? (Ideally include a list of successfully achieved goals.)
\item What is the quality of the result? Is the problem solved for good and the mankind does not need to ever think about it again, or just partially improved upon? (Is the incompleteness caused by overwhelming problem complexity that would be out of thesis scope\todo{This is quite common.}, or any theoretical reasons, such as computational hardness?)
\item Does the result have any practical applications that improve upon something realistic?
\item Is there any good future development or research direction that could further improve the results of this thesis? (This is often summarized in a separate subsection called `Future work'.)
\end{itemize}


\ifEN
\chapwithtoc{Bibliography}
\else
\chapwithtoc{Seznam použité literatury}
\fi

\printbibliography[heading=none]


\appendix
\chapter{Using CoolThesisSoftware}

Use this appendix to tell the readers (specifically the reviewer) how to use your software. A very reduced example follows; expand as necessary. Description of the program usage (e.g., how to process some example data) should be included as well.

To compile and run the software, you need dependencies XXX and YYY and a C compiler. On Debian-based Linux systems (such as Ubuntu), you may install these dependencies with APT:
\begin{Verbatim}
apt-get install \
  libsuperdependency-dev \
  libanotherdependency-dev \
  build-essential
\end{Verbatim}

To unpack and compile the software, proceed as follows:
\begin{Verbatim}
unzip coolsoft.zip
cd coolsoft
./configure
make
\end{Verbatim}

The program can be used as a C++ library, the simplest use is demonstrated in \cref{lst:ex}. A demonstration program that processes demonstration data is available in directory \verb|demo/|, you can run the program on a demonstration dataset as follows:
\begin{Verbatim}
cd demo/
./bin/cool_process_data data/demo1
\end{Verbatim}

After the program starts, control the data avenger with standard \verb-WSAD- controls.

\begin{listing}
\begin{lstlisting}
#include <CoolSoft.h>
#include <iostream>

int main() {
	int i;
	if(i = cool::ProcessAllData()) // returns 0 on error
		std::cout << i << std::endl;
	else
		std::cerr << "error!" << std::endl;
	return 0;
}
\end{lstlisting}
\caption{Example program.}
\label{lst:ex}
\end{listing}


% if your attachments are complicated, describe them in a separate appendix
%\include{attachments}

\openright
\end{document}
