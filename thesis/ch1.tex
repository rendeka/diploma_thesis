\chapter{Skyrmion}
\label{chap1}

\section{Magnetic skyrmions}

Magnetic skyrmions are 2-D topologically stable spin formations

The essential role in the formation of magnetic skyrmions plays Dzyaloshinskii–Moriya (DM) interaction (DMI), also called antisymetric exchange. 

\xxx{Wiki: The discovery of antisymmetric exchange originated in the early 20th century from the controversial observation of weak ferromagnetism in typically antiferromagnetic Fe2O3 crystals.[1] In 1958, Igor Dzyaloshinskii provided evidence that the interaction was due to the relativistic spin lattice and magnetic dipole interactions based on Lev Landau's theory of phase transitions of the second kind.[2] In 1960, Toru Moriya identified the spin-orbit coupling as the microscopic mechanism of the antisymmetric exchange interaction.[1] Moriya referred to this phenomenon specifically as the "antisymmetric part of the anisotropic superexchange interaction." The simplified naming of this phenomenon occurred in 1962, when D. Treves and S. Alexander of Bell Telephone Laboratories simply referred to the interaction as antisymmetric exchange. Because of their seminal contributions to the field, antisymmetric exchange is sometimes referred to as the Dzyaloshinskii–Moriya interaction.}

It is a magnetic exchange interaction between neighbouring patricles with spin. It can be charecterized by hamiltonian:

\begin{equation}
	\label{DM-interaction-hamiltonian}
	H^{\text{DM}}_{ij} = \bm{D}_{ij} \dot (\bm{S}_i \times \bm{S}_j),
\end{equation}

where $\bm{D}_{ij}$ is a Dzyaloshinskii–Moriya vector characterizing the interaction, and $\bm{S}_i$, $\bm{S}_j$ are spin vectors. 

This means that the DM interaction prefers perpendicular orientation of neighboring spins. The vector $\bm{D}_{ij}$ is given by the symmetry of the system. DMI mediates also one other stable magnetic arrangement which is a \emph{spin spiral}. 

The direction of vector $\bm{D}_{ij}$ is dictated by the system symmetry, and influences the topology of skyrmions and spirals.

Main source \cite{li2023magnetic}



\section{Dataset}

\todo{how do we get our data}

Defining part of any machine learning problem is the dataset we are working with. Our dataset consists of in total 9840 images of magnetic configurations created by Monte-Carlo simulations on discrete 200x200 square grid with periodic boundry conditions near absolute zero temperature. The simulations were done in conditions given be values of $D$ and $B$. Used values of $D$ ranged from We have 5 simulated different configurations for each pair of $D$ and $B$ values.


\todo{chalenges of this dataset}

\todo{train dataset}

\todo{one-hot encoding for labels}

